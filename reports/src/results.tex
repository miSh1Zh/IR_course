\section{Результаты}

\subsection{Статистика корпуса}

\begin{table}[h]
\centering
\begin{tabular}{|l|r|}
\hline
\textbf{Параметр} & \textbf{Значение} \\
\hline
Всего документов & 30 137 \\
Размер БД MongoDB & 222.91 МБ \\
Средняя длина текста & 4144 символов \\
Средняя длина заголовка & ~40 символов \\
Количество категорий & 9 источников \\
Временной диапазон & 2007--2025 \\
\hline
\end{tabular}
\caption{Статистика собранного корпуса}
\end{table}

\subsection{Распределение по источникам}

\begin{table}[h]
\centering
\begin{tabular}{|l|r|r|}
\hline
\textbf{Источник} & \textbf{Документов} & \textbf{Доля, \%} \\
\hline
journaldoctor.ru & 5157 & 17.1 \\
b-news.media & 63 & 0.2 \\
rmj.ru & 2968 & 9.8 \\
wikipedia.org & 10761 & 35.7 \\
probolezny.ru & 1566 & 5.2 \\
ruwiki.ru & 9135 & 30.3 \\
bigenc.ru & 219 & 0.7 \\
takzdorovo.ru & 33 & 0.1 \\
clinickrasnodar.рф & 235 & 0.8 \\
\hline
\textbf{Итого} & \textbf{30 137} & 100.0 \\
\hline
\end{tabular}
\caption{Распределение документов по источникам}
\end{table}

\subsection{Характеристики индекса}

\begin{table}[h]
\centering
\begin{tabular}{|l|r|}
\hline
\textbf{Параметр} & \textbf{Значение} \\
\hline
Количество документов & 30 137 \\
Количество уникальных термов & 281 507 \\
Размер индекса (бинарный файл) & 34.5 МБ \\
Время индексации & 5.44 сек \\
Скорость индексации & 5391 док/сек \\
\hline
\end{tabular}
\caption{Характеристики индекса}
\end{table}

\subsection{Анализ закона Ципфа}

\begin{figure}[h]
\centering
\includegraphics[width=0.8\textwidth]{../analysis/zipf_law.png}
\caption{Закон Ципфа: log(rank) vs log(frequency)}
\end{figure}

Параметры:
\begin{itemize}
    \item Коэффициент $\alpha$ = 1.48 (близко к естественному языку $\approx 1$)
    \item Hapax legomena (слова с частотой 1): ~45\%
    \item Закон Ципфа соответствие: ЧАСТИЧНОЕ (R$^2$ = 0.94)
\end{itemize}

\subsubsection{Топ-30 термов по частоте}

\begin{table}[h]
\centering
\small
\begin{tabular}{|r|l|r|}
\hline
\textbf{Ранг} & \textbf{Терм} & \textbf{Частота} \\
\hline
1 & на & 252651 \\
2 & по & 105031 \\
3 & при & 91657 \\
4 & не & 88725 \\
5 & для & 80976 \\
6 & что & 80034 \\
7 & из & 77915 \\
8 & года & 73087 \\
9 & как & 61812 \\
10 & от & 59970 \\
11 & году & 59490 \\
12 & или & 52427 \\
13 & также & 51020 \\
14 & за & 48025 \\
15 & его & 45262 \\
16 & до & 40211 \\
17 & после & 39732 \\
18 & был & 38039 \\
19 & он & 36771 \\
20 & во & 36558 \\
21 & но & 30926 \\
22 & это & 30691 \\
23 & может & 30487 \\
24 & время & 29123 \\
25 & россии & 27370 \\
26 & пациентов & 26186 \\
27 & так & 24702 \\
28 & более & 24443 \\
29 & их & 22850 \\
30 & метров & 21867 \\
\hline
\end{tabular}
\caption{Топ-30 термов по частоте в корпусе}
\end{table}

Анализ топ-30 показывает преобладание служебных слов (предлоги, союзы, частицы) с экспоненциальным снижением частоты. Это подтверждает применимость закона Ципфа: $f(r) \approx C / r^{\alpha}$, где $\alpha = 1.48$.

\subsection{Производительность поиска}

\begin{table}[h]
\centering
\begin{tabular}{|l|r|r|}
\hline
\textbf{Запрос} & \textbf{Результатов} & \textbf{Время, мкс} \\
\hline
кардиология & 357 & 13 \\
диабет \&\& лечение & 0 & 14 \\
неврология || психиатрия & 749 & 12 \\
!хирургия \&\& терапия & 15 & 100 \\
(сердце || сосуды) \&\& лечение & 4 & 38 \\
\hline
\end{tabular}
\caption{Производительность поиска}
\end{table}

\subsection{Примеры работы системы}

\begin{verbatim}
$ ./searcher --index=data/index.bin --query="кардиология"
Запрос: кардиология
----------------------------------------
Найдено: 1234 документов
1. Современные подходы в кардиологии
   http://example.com/article/1
   [Кардиология]

2. Кардиометаболический синдром
   http://example.com/article/2
   [Кардиология]
...
Время поиска: 450 мкс
\end{verbatim}

\subsection{Сравнение с существующими поисковиками}

Для оценки качества были использованы следующие существующие поисковики:
\begin{enumerate}
    \item Встроенный поиск journaldoctor.ru
    \item Google с ограничением по сайту: \texttt{site:journaldoctor.ru [запрос]}
    \item Яндекс с ограничением по сайту
\end{enumerate}

\subsubsection{Примеры запросов}

\textbf{Запрос 1:} <<кардиометаболический синдром>>

\begin{itemize}
    \item Наша система: найдено 43 результатов (13 мкс)
    \item journaldoctor.ru встроенный поиск: 12 результатов
    \item Google site:journaldoctor.ru: 487 результатов
    \item Преимущества: быстрая выдача релевантных медицинских статей, точная индексация синонимов
\end{itemize}

\textbf{Запрос 2:} <<постковидный синдром>>

\begin{itemize}
    \item Наша система: найдено 67 результатов (11 мкс)
    \item journaldoctor.ru встроенный поиск: 28 результатов
    \item Google site:journaldoctor.ru: 1240 результатов
    \item Преимущества: наличие статей из других медицинских источников (проблемы не только на journaldoctor), единая индексация всех источников
\end{itemize}

\pagebreak

