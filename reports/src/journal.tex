\section{Журнал выполнения}

В данном разделе описываются проблемы, возникшие в процессе выполнения работы, и выбранные методы их решения.

\subsection{Проблема 1: Определение структуры сайта}

\textbf{Описание:} При начале работы с journaldoctor.ru потребовалось определить структуру URL'ов и CSS-селекторы для извлечения контента.

\textbf{Решение:} 
\begin{itemize}
    \item Изучена HTML-структура страниц с помощью инструментов разработчика браузера
    \item Определены селекторы для заголовков, текста статей, категорий
    \item Реализован универсальный подход с несколькими альтернативными селекторами
\end{itemize}

\subsection{Проблема 2: Сохранность данных при перезапуске}

\textbf{Описание:} При остановке Docker-контейнера данные MongoDB терялись.

\textbf{Решение:} 
Настроены Docker volumes для персистентного хранения:
\begin{verbatim}
volumes:
  mongo_data:
    driver: local
\end{verbatim}

\subsection{Проблема 3: Дубликаты документов}

\textbf{Описание:} При повторном запуске краулера добавлялись дубликаты статей.

\textbf{Решение:} 
Создан уникальный индекс по URL в MongoDB:
\begin{verbatim}
self.collection.create_index('url', unique=True)
\end{verbatim}

\subsection{Проблема 4: Rate limiting}

\textbf{Описание:} При быстром сборе данных сервер мог заблокировать краулер.

\textbf{Решение:} 
Настроены параметры вежливого краулинга:
\begin{verbatim}
ROBOTSTXT_OBEY = True
DOWNLOAD_DELAY = 2
RANDOMIZE_DOWNLOAD_DELAY = True
CONCURRENT_REQUESTS_PER_DOMAIN = 1
\end{verbatim}

\subsection{Проблема 5: Реализация без STL}

\textbf{Описание:} Требовалось реализовать хеш-таблицу без использования \texttt{std::unordered\_map}.

\textbf{Решение:} 
Реализована собственная хеш-таблица с методом цепочек, автоматическим перехешированием и поддержкой шаблонов.

\subsection{Проблема 6: Обработка UTF-8}

\textbf{Описание:} Необходимо корректно обрабатывать кириллицу в токенизаторе и стеммере.

\textbf{Решение:} 
Реализованы функции для работы с UTF-8 символами: определение типа символа, приведение к нижнему регистру с учётом кириллицы.

\pagebreak

