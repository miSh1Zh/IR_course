\section{Результаты работы}

В рамках курса «Информационный поиск» разработана полнофункциональная система поиска по медицинским статьям:

\begin{enumerate}
    \item Разработан веб-краулер на базе Scrapy для сбора медицинских статей из 9 источников.
    
    \item Создана инфраструктура на Docker с персистентным хранением данных в MongoDB.
    
    \item Собран корпус из 30 137 документов общим объёмом 222.91 МБ по медицине и связанным с ней темами на русском языке.
    
    \item Реализован токенизатор с поддержкой UTF-8 (латиница + кириллица).
    
    \item Реализован стеммер для русского и английского языков.
    
    \item Разработана собственная реализация хеш-таблицы с методом цепочек и автоматическим перехешированием.
    
    \item Построен инвертированный индекс с сохранением в бинарном формате.
    
    \item Проведён анализ закона Ципфа --- распределение соответствует теоретическому с $\alpha \approx 1$.
    
    \item Реализован булевый поиск с поддержкой операций AND, OR, NOT и скобок.
    
    \item Разработаны unit-тесты и интеграционные тесты.
    
    \item Проведён сравнительный анализ с существующими поисковиками.
\end{enumerate}

\subsection{Оценка качества работы}

\textbf{Достоинства реализации:}
\begin{itemize}
    \item Модульная архитектура позволяет легко добавлять новые источники
    \item Персистентное хранение защищает от потери данных
    \item Автоматическая дедупликация документов
    \item Соблюдение robots.txt и вежливый краулинг
    \item Реализация без использования сложных структур STL (только vector и string)
    \item Эффективные алгоритмы булевых операций на отсортированных списках
    \item Поддержка UTF-8 для работы с кириллицей
\end{itemize}

\textbf{Недостатки и ограничения:}
\begin{itemize}
    \item Зависимость от структуры сайтов-источников (при изменении вёрстки потребуется доработка селекторов)
    \item Отсутствие полнотекстового контента для некоторых статей (только аннотации)
    \item Упрощённый стемминг может давать неточные результаты
    \item Отсутствие ранжирования результатов (TF-IDF, BM25)
    \item Нет поддержки фразового поиска
\end{itemize}

% \subsection{Выполнение требований}

% \begin{tabular}{|p{8cm}|c|}
% \hline
% \textbf{Требование} & \textbf{Выполнено} \\
% \hline
% Корпус 30,000+ документов & Да \\
% Минимум 3 источника & Да \\
% Уникальность корпуса & Да \\
% Наличие существующих поисковиков & Да \\
% Статистический анализ & Да \\
% Реализация без STL (кроме vector/string) & Да \\
% Токенизация & Да \\
% Стемминг & Да \\
% Собственная хеш-таблица & Да \\
% Инвертированный индекс & Да \\
% Анализ закона Ципфа & Да \\
% Булев поиск (AND, OR, NOT, скобки) & Да \\
% Unit-тесты & Да \\
% Интеграционные тесты & Да \\
% \hline
% \end{tabular}

\subsection{Направления развития}

Возможные улучшения системы:

\begin{enumerate}
    \item Добавить автоматическое обновление корпуса по расписанию (cron)
    \item Реализовать извлечение дополнительных метаданных (авторы, ключевые слова, DOI)
    \item Добавить мониторинг состояния краулера
    \item Расширить корпус до 1M+ документов (для максимальной оценки)
    \item Реализовать дополнительные источники данных
    \item Добавить сжатие posting lists (delta + Variable Byte Encoding)
    \item Реализовать кэширование частых запросов
    \item Добавить ранжирование результатов (TF-IDF, BM25)
    \item Реализовать фразовый поиск (позиционный индекс)
    \item Добавить поддержку wildcard queries
    \item Реализовать исправление опечаток (spell checking)
    \item Добавить подсветку результатов в тексте
\end{enumerate}

\subsection{Список литературы}

\begin{enumerate}
    \item Manning C.D., Raghavan P., Schütze H. Introduction to Information Retrieval. Cambridge University Press, 2008.
    \item Porter M.F. An algorithm for suffix stripping. Program, 14(3):130--137, 1980.
    \item Zipf G.K. Human Behavior and the Principle of Least Effort. Addison-Wesley, 1949.
\end{enumerate}

