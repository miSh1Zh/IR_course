\section{Тестирование системы}

Для обеспечения качества и корректности работы системы разработан комплекс тестов, покрывающий все основные компоненты.

\subsection{Тестирование краулера}

\subsubsection{Тест 1: Корректность извлечения данных}
\textbf{Цель:} Проверить, что spider корректно извлекает все поля статьи.

\textbf{Шаги:}
\begin{enumerate}
    \item Запустить spider на ограниченном количестве статей (10 штук)
    \item Проверить наличие всех полей: title, text, url, category, year
    \item Проверить длину текста (должна быть > 100 символов)
\end{enumerate}

\textbf{Ожидаемый результат:} Все поля заполнены, текст не пустой.

\subsubsection{Тест 2: Обработка дубликатов}
\textbf{Цель:} Проверить, что дубликаты не добавляются в базу.

\textbf{Шаги:}
\begin{enumerate}
    \item Записать количество документов в базе
    \item Запустить spider повторно на тех же страницах
    \item Проверить количество документов
\end{enumerate}

\textbf{Ожидаемый результат:} Количество документов не изменилось.

\subsubsection{Тест 3: Работа с Unicode}
\textbf{Цель:} Проверить корректную обработку кириллицы.

\textbf{Шаги:}
\begin{enumerate}
    \item Собрать статьи на русском языке
    \item Проверить корректность заголовков и текста
    \item Проверить экспорт в JSON
\end{enumerate}

\textbf{Ожидаемый результат:} Кириллица отображается корректно.

\subsection{Unit-тесты поискового движка}

Реализованы unit-тесты для всех компонентов системы:

\subsubsection{Тесты токенизатора}
\begin{itemize}
    \item \texttt{test\_tokenizer\_simple} --- базовая токенизация
    \item \texttt{test\_tokenizer\_punctuation} --- обработка знаков препинания
    \item \texttt{test\_tokenizer\_russian} --- кириллица
    \item \texttt{test\_tokenizer\_mixed} --- смешанный текст
    \item \texttt{test\_tokenizer\_empty} --- пустой ввод
\end{itemize}

\subsubsection{Тесты стемминга}
\begin{itemize}
    \item \texttt{test\_stemmer\_english} --- английские слова
    \item \texttt{test\_stemmer\_russian} --- русские слова
    \item \texttt{test\_stemmer\_short} --- короткие слова
\end{itemize}

\subsubsection{Тесты HashMap}
\begin{itemize}
    \item \texttt{test\_hashmap\_insert\_find} --- вставка и поиск
    \item \texttt{test\_hashmap\_update} --- обновление значения
    \item \texttt{test\_hashmap\_operator} --- оператор []
    \item \texttt{test\_hashmap\_many} --- 10,000 элементов (проверка rehash)
\end{itemize}

\subsubsection{Тесты булевых операций}
\begin{itemize}
    \item \texttt{test\_intersect} --- пересечение
    \item \texttt{test\_intersect\_empty} --- пустой результат
    \item \texttt{test\_union} --- объединение
    \item \texttt{test\_union\_empty} --- объединение с пустым списком
\end{itemize}

\subsubsection{Тесты парсера запросов}
\begin{itemize}
    \item \texttt{test\_query\_parser\_simple} --- одно слово
    \item \texttt{test\_query\_parser\_and} --- явное AND
    \item \texttt{test\_query\_parser\_or} --- OR
    \item \texttt{test\_query\_parser\_not} --- NOT
    \item \texttt{test\_query\_parser\_complex} --- сложный запрос со скобками
    \item \texttt{test\_query\_parser\_implicit\_and} --- неявное AND (пробел)
\end{itemize}

\subsection{Интеграционные тесты}

Скрипт \texttt{integration\_test.sh} проверяет работу системы в целом:
\begin{enumerate}
    \item Создание тестового корпуса (5 документов)
    \item Построение индекса
    \item Поиск по одному слову
    \item Поиск с AND
    \item Поиск с OR
    \item Поиск с NOT
    \item Проверка пустого результата
\end{enumerate}

\subsection{Результаты тестирования}

Все unit-тесты (22 теста) и интеграционные тесты проходят успешно:

\begin{verbatim}
$ make test
./test_runner
========================================
      UNIT ТЕСТЫ
========================================

--- Тесты токенизатора ---
test_tokenizer_simple
test_tokenizer_punctuation
test_tokenizer_russian
test_tokenizer_mixed
test_tokenizer_empty

--- Тесты стемминга ---
test_stemmer_english
test_stemmer_russian
test_stemmer_short

--- Тесты HashMap ---
test_hashmap_insert_find
test_hashmap_update
test_hashmap_operator
test_hashmap_many

--- Тесты булевых операций ---
test_intersect
test_intersect_empty
test_union
test_union_empty

--- Тесты парсера запросов ---
test_query_parser_simple
test_query_parser_and
test_query_parser_or
test_query_parser_not
test_query_parser_complex
test_query_parser_implicit_and

========================================
      ВСЕ ТЕСТЫ ПРОЙДЕНЫ!
========================================
\end{verbatim}

\pagebreak

